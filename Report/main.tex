\documentclass[a4paper,10pt]{article}
\usepackage[margin=2.5cm]{geometry}
\usepackage[utf8]{inputenc}
\usepackage[colorlinks=true,urlcolor=blue]{hyperref}
\usepackage{amsmath}
\usepackage{graphicx}
\usepackage{float}
\usepackage{caption}
\usepackage{subcaption}

%\usepackage{listings} %Alternative to minted
\usepackage{minted}



\setlength{\parindent}{0em}
\setlength{\parskip}{1em}

\title{\textbf{1RT730 Project Report} 
	\\ Harald - The ultimate Swedish chatbot}
\author{Joel Sundin, Petter Möllerström, Rahul Sebastian Peter}
\date{\today}

\begin{document}
	
\maketitle
    
 \section{Introduction}
This project presents a Swedish language learning application powered by Large Language Models (LLMs). The motivation behind developing this application is to make Swedish learning more interactive, personalized and accessible for non-native speakers. Traditional learning methods often lack engagement and real-time feedback making it difficult for learners to practice comprehension and conversational skills naturally. The proposed system addresses this challenge by combining automated reading and writing comprehension exercises, vocabulary quizzes, practicing weak words using flashcards and an AI-driven tutor that adapts to the learner’s proficiency level. In a societal context, this application promotes language inclusivity and supports immigrants, students, and professionals integrating into Swedish society. The main contribution of this work is the integration of LLM-based interaction into a structured learning environment that enables personalized practice, contextual understanding and targeted feedback.


\section{Dataset}
In this section, describe the dataset(s) that support your LLM application. Specify their source, size, and type (e.g., text, images, multimodal data). Discuss any preprocessing or data-cleaning steps applied, aand methods used to prepare the data for training, fine-tuning, or evaluation. If the application relies on prompt engineering or synthetic data generation instead of a formal dataset, explain the rationale and design choices. Highlight any limitations, biases, or challenges present in the dataset, and justify why it is suitable for your application.

\section{Architecture}
This section should outline the overall design and workflow of the LLM-based application. Present the system architecture, describing how inputs are processed, how the LLM is integrated, and how outputs are generated and refined. If applicable, explain supporting components such as retrieval-augmented generation (RAG) modules, safety filters, knowledge bases, or user interfaces. Visual aids such as figures or flowcharts are encouraged to illustrate the data flow and interactions between components. Also please justify the design choices and explain how the architecture supports the application's goals.

\section{Results}
In this section, present the outcomes of your experiments or evaluations. Provide both quantitative results (e.g., accuracy, precision, recall, BLEU, latency) and qualitative analyses (e.g., case studies, user feedback, example outputs). Where possible, compare the performance of your system against baseline methods or alternative approaches. Discuss the strengths of your application as well as its limitations, including cases where it may fail or underperform. Use tables, graphs, or figures to support clarity and emphasize key findings.

\section{Societal Impact}
Here, reflect on the broader implications of your LLM application for society. Discuss potential benefits such as improved accessibility, efficiency, education, or democratization of knowledge. At the same time, critically examine possible risks, including issues of bias, fairness, privacy, misuse, or environmental impact. Consider ethical challenges and describe mitigation strategies, such as transparency, safety mechanisms, or human oversight. This section should demonstrate awareness of the responsibility associated with deploying LLM-based systems in real-world contexts.

	
\hfill \break
\textit{\textbf{Use of generative AI}}

Write a few sentences if you have or have not used any generative AI, and if so how.


\hfill \break
\textit{\textbf{Note}}

Report should be max. 4-6 pages long.

	
\begin{thebibliography}{1}

	%\bibitem{Item} Add here.
		
\end{thebibliography}

	
	
%	\pagebreak
%	
%	\appendix
%	\section{Code}
%	\label{app:code}
%	
%	\inputminted[frame=lines,framesep=2mm,linenos]{python}{your_code.py}
%	
\end{document}